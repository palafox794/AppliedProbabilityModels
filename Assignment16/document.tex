\documentclass[letterpaper, 10 pt, conference]{article} 
\usepackage[spanish]{babel}
\usepackage{amsmath,amssymb,amscd,amsthm} % variety of useful math macros
\usepackage[inner=1.5 cm, outer = 1.5 cm, top=1 cm, bottom = 1.5 cm]{geometry}
\usepackage{subcaption}
%For inserting graphics
\usepackage{graphicx}
\usepackage[dvipsnames]{xcolor}
\usepackage{listings}
\usepackage[utf8]{inputenc}
\usepackage{hyperref}
\usepackage{array, multirow}
\usepackage{lipsum}
\usepackage{natbib}
\usepackage[normalem]{ulem}


\bibliographystyle{abbrvnat}

\newtheorem{thm}{Theorem}
\newtheorem{prop}{Proposition}
\newtheorem{lemma}{Lemma}
\newtheorem{ex}{Exercise}
\newtheorem{exm}{Example}
\newtheorem{defn}{Definition}

\newcommand\E{\ensuremath{\mathbb{E}}}
\newcommand\N{\ensuremath{\mathbb{N}}}
\renewcommand{\P}{\ensuremath{\mathbb{P}}}
\newcommand\Q{\ensuremath{\mathbb{Q}}}
\newcommand\R{\ensuremath{\mathbb{R}}}
\newcommand\Z{\ensuremath{\mathbb{Z}}}

\newcommand\var[1]{\, \mathrm{Var} \left( #1 \right)}

\newcommand\pr[1]{\, \mathbb{P} \left\lbrace #1 \right\rbrace}

\newcommand\cov[1]{\, \mathrm{Cov} \left( #1 \right)}

\newcommand\expec[1]{\, \mathbb{E} \left\lbrack #1 \right\rbrack}

\title{Reseñas a proyectos
}
\date{15 de diciembre del 2020}

\hypersetup{
	colorlinks=true,
	linkcolor=blue,
	filecolor=magenta,      
	urlcolor=blue,
	citecolor=MidnightBlue
}

\author{Gerardo {\textsc{Palafox}}}

\begin{document}
\maketitle

Las siguientes son observaciones que se le hicieron a compañeros del posgrado respecto a sus proyectos. 

\begin{description}
	\item[A Fabiola, sobre modelado multi agente de epidemias.] El proyecto se ve bastante víable e interesante. Sería de provecho quizá comparar los resultados de este modelo con otro tipo de planteamientos, como las epidemias modeladas con cadenas de Markov u otros procesos estocásticos. Así mismo, explorar las capacidades del enfoque multiagente para epidemias con poblaciones no homogeneas.
	
	\item[A Fabiola, sobre redes bayesianas.] No me queda claro en qué aspecto específico de la economía piensas aplicar el concepto de red bayesiana, ni de qué forma. Si bien son una herramienta fuerte, si no está claro en qué precisamente las usarás, te puedes perder a la hora de llevar el proyecto. 
	
	\item[A Gabriela, sobre análisis de índices delictivos.] Me parece algo general el hacer estadística descriptiva a los datos de inseguridad. ¿Tienes algunas ideas sobre qué específicamente buscas encontrar? Por ejemplo, las distintas tendencias delictivas que haya en las diversas regiones del país. Una idea que a mi me parecería  interesante es buscar alguna correlación de estas tendencias con parámetros de índole social, como los que suele proporcionar el INEGI.
	
\end{description}

%%%%%%%%%%%%%%%%%%%%%%%%%%%%%%%%%%%%%%%%%%%%%%%%%%%%%%%%%%%%%%%%%%%%%%%%%%%%%%%


\end{document}
